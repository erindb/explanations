\documentclass[10pt,letterpaper]{article}

\usepackage{hyperref, breakurl}
\usepackage{cogsci}
\usepackage{pslatex}
\usepackage{apacite}
\usepackage{graphicx}
\usepackage{caption}
\usepackage{subcaption}
\usepackage{color}
\usepackage{amsfonts}
\usepackage{amsmath}

\newcommand{\w}[1]{\emph{#1}}
\newcommand{\todo}[1]{{\color{red}#1}}
\newcommand{\ndg}[1]{{\color{green}#1}}

\title{Explanations in context}
 
\author{{\large \bf Erin Bennett} (erindb@stanford.edu), {\large \bf Noah D.~Goodman} (ngoodman@stanford.edu)\\
  Department of Psychology, Stanford University.}
  
\begin{document}

\maketitle

\begin{abstract}

Proposal: 250 words

Different factors seem to affect the selection of the best explanation from a set of possible causes. Some of these factors seem to be the result of pragmatic inference. We implement an existing probabilistic model of counterfactual reasoning to identify possible causes (and gradable judgements of how causally relevant each might be) \cite{LucasKemp2015}, along with a model of rational communication \cite{FrankGoodman2012, GoodmanStuhlmuller2013} to investigate how changes to the background causal model, changes to what information is common knowledge, and changes to the question under discussion (QUD) can affect people's judgments of how good a potential cause is at explaining an effect. In particular, we test such a model in situations where interlocuters have different knowledge states to see whether it tracks people's judgments of the goodness of explanations, and also look at whether the effect of moral judments on causal selection can be explained by QUD effects.

\textbf{Keywords:} 
explanations; counterfactuals; pragmatics
\end{abstract}

\section{Background}

\subsection{Causal Selection}

\citeA{Hesslow1988} discusses the issue of \emph{causal selection} as a seperate task from \emph{causal attribution}.
The latter determines whether a condition can truthfully be considered a contributing cause.
The former determines, of possible contributing causes, which is the best \emph{explanation}.

Hesslow claims that an explanation must always explain why an effect occurred in the actual situation but did not occur (or would not have occurred) in some contextually-specified comparison class of situations.
He uses this account to unify many previously proposed criteria for causal selection. %: predictability, temporal proximity, abnormality, stability, deviation from a theorical ideal, responsibility, sufficiency, neccesity, replacibility, efficacy, and interest.
The relevant comparison class must depend on the interlocuters' knowledge of what alternative situations are possible, and on the topic of their conversation.
Under some topics, the utility, moral acceptability, temporal proximity, typicality or stability of the different alternatives might matter to the generation of the comparison class.

\citeA{Hilton1996} desribes an informal model of Gricean pragmatics in explanations.
%and empirically verifies some predictions of that model.
He explains between casual selction processes of causal ``discounting'' and causal ``backgrounding'' in terms of different Gricean maxims.
Causal discounting, where one cause becomes a less good explanation as a result of an alternative cause gaining salience, seems to be an effect of changes in the underlying generative model (\emph{quality}, or truthfulness) or in the question under discussion (\emph{relevance}).
Causal backgrounding, where a cause becomes a less good explanation as a result of becoming especially predictable, seems to happen as a result of changes what common knowledge is assumed between the interlocuters (\emph{informativity}).

\citeA{ReuterEtAl2014} describe factors that seem to guide causal selection that they consider to be independent of pragmatic effects or the underlying causal structure: temporal proximity and morality. They show through a series of experiments that the most recent potential causes A and B is regarded as the best explanation of an effect E when the causal structure is ``A and B implies E'', and that this effect is overridden by a tendency to hold as responsible any cause that violates a norm (e.g. if A broke a rule, and B did not, then A will be held responsible for the effect jointly caused by both A and B).

\subsection{Counterfactuals}

\citeA{LucasKemp2015} ...

\subsection{Rational Speech Act Models}

RSA models \cite{FrankGoodman2012, GoodmanStuhlmuller2013} and related models \cite{Franke2011, Russell2012}.

\bibliographystyle{apacite}

\setlength{\bibleftmargin}{.125in}
\setlength{\bibindent}{-\bibleftmargin}

\bibliography{explanations}

\end{document}
