\documentclass[10pt,letterpaper]{article}

\usepackage{hyperref, breakurl}
\usepackage{cogsci}
\usepackage{pslatex}
\usepackage{apacite}
\usepackage{graphicx}
\usepackage{caption}
\usepackage{subcaption}
\usepackage{color}
\usepackage{amsfonts}
\usepackage{amsmath}

\newcommand{\w}[1]{\emph{#1}}
\newcommand{\todo}[1]{{\color{red}#1}}
\newcommand{\ndg}[1]{{\color{green}#1}}

\title{Explanations in context}
 
\author{{\large \bf Erin Bennett} (erindb@stanford.edu), {\large \bf Noah D.~Goodman} (ngoodman@stanford.edu)\\
  Department of Psychology, Stanford University.}
  
\begin{document}

\maketitle

\begin{abstract}

Proposal: 250 words

\textbf{Keywords:} 
explanations; counterfactuals; pragmatics
\end{abstract}

\section{Background}

\subsection{Causal Selection}

\citeA{Hesslow1988} discusses the issue of \emph{causal selection} as a seperate task from \emph{causal attribution}.
The latter determines whether a condition can truthfully be considered a contributing cause.
The former determines, of possible contributing causes, which is the best \emph{explanation}.

Hesslow claims that an explanation must always explain why an effect occurred in the actual situation but did not occur (or would not have occurred) in some contextually-specified comparison class of situations.
He uses this account to unify many previously proposed criteria for causal selection. %: predictability, temporal proximity, abnormality, stability, deviation from a theorical ideal, responsibility, sufficiency, neccesity, replacibility, efficacy, and interest.
%Hesslow shows how several different previously proposed criteria for causal selection %(e.g. abnormality, deviation from an ideal, temporal proximity, prior probability)
%can be summarized: An explanation explains why the effect occurred in the actual situation but did not occur (or would not have occurred) in some contextually-specified comparison class of situations.
The relevant comparison class must depend on the interlocuters' knowledge of what alternative situations are possible, and on the topic of their conversation.
Under some topics, the utility, moral acceptability, temporal proximity, typicality or stability of the different alternatives might matter to the generation of the comparison class.
%since the relevant comparison class might be affected by the interlocuter's interest in discussing 
%various contextual factors (i.e. by
%the utility, moral acceptability, typicality or stability of the different alternatives.%, or by the  
%question under discussion, or the
%their knowledge of what alternatives are possible.
%In turn, the question under discussion might be affected by utility, moral acceptability, typicality, or stability of different alternative situations.
%The knowledge states of the interlocuters might be affected by their interest levels.
%The relevant comparison class for an explanation might be affected by the prior probabilities, utilities, or moral acceptabilities of different alternative situations.

\bibliographystyle{apacite}

\setlength{\bibleftmargin}{.125in}
\setlength{\bibindent}{-\bibleftmargin}

\bibliography{explanations}

\end{document}
